% Dokumententyp (Schriftgröße, Papiergröße, Formelausrichtung)
\documentclass[pdftex,12pt,a4paper,fleqn]{scrartcl}
 
% Ein- und Ausgabekodierung
\usepackage[utf8]{inputenc}
\usepackage[T1]{fontenc}

% Sprache
\usepackage[ngerman]{babel}
\usepackage[babel,german=quotes]{csquotes}

% Schriftart
\usepackage{lmodern}

% Euro-Symbol
\usepackage{eurosym}

% Tabellen
\usepackage{longtable}
\usepackage{array}

% PDF-Seiten Landscape
\usepackage{pdflscape}

% Schlussregeln
\usepackage{proof}

% Listen
\usepackage{enumitem}

% Formelsatz
\usepackage[fleqn]{amsmath}
\usepackage{amsfonts}
\usepackage{amssymb}

% Zitate
\usepackage[backend=biber,style=alphabetic]{biblatex}

% Hurenkinder- und Schusterjungenregelung
\clubpenalty = 10000
\widowpenalty = 10000
\displaywidowpenalty = 10000

% Variablen
\newcommand{\doctitle}{MFI II}
\newcommand{\docsubtitle}{CheatSheet}
\newcommand{\docauthor}{Marvin Hofmann}
\newcommand{\docdate}{\today}

% PDF-Links
\usepackage[ngerman,pdfauthor={\docauthor},
pdftitle={\doctitle},
colorlinks=true,linkcolor=black,citecolor=black,urlcolor=blue,
filecolor=black]{hyperref}

% Titelseite
\title{\doctitle}
\subtitle{\docsubtitle}
\author{\docauthor}
\date{\docdate}

% Zeilenabstand
\usepackage{setspace}

% Seiten-Geometrie
\usepackage[paper=a4paper,left=2.5cm,right=2.5cm,bottom=4.5cm,top=3.5cm]{geometry}

% Kopf- und Fußzeilen
\usepackage[automark]{scrpage2}
\ihead[]{\parbox[c][1.5cm][c]{0.5\textwidth}{{\large \textsf{\textbf{\doctitle}}} \\ \docsubtitle}}
\ohead[]{\parbox[c][1.5cm][c]{0.5\textwidth}{\raggedleft \begin{tabular}{r} \docauthor\end{tabular}}}
\ifoot[]{}
\cfoot[]{\thepage}
\ofoot[]{}
\pagestyle{scrheadings}
\setheadtopline{.5pt}
\setheadsepline{.5pt}
\setlength{\headheight}{1.5cm}

% Fußnoten
\usepackage[hang]{footmisc}
\setlength{\footnotemargin}{3mm}

% Todonotes
\usepackage{todonotes}

% Boxen
\usepackage[framemethod=tikz]{mdframed}

% Grafiken und Fotos
\usepackage{tikz}
\usepackage{tikz-qtree}
\usepackage{circuitikz}
\usetikzlibrary{decorations,calc,trees,positioning,arrows,fit,shapes,calc}
\usepackage{graphicx}
\usepackage{wrapfig}
\usepackage{caption}
\usepackage{subcaption}

% Farben
\usepackage{color}

% Floating
\usepackage{float}

% Formatierung der Paragraphen
\parindent 0pt
\parskip 6pt

% Euro- und Gradzeichen (Unicode)
\DeclareUnicodeCharacter{B0}{$^{\circ}$}
\DeclareUnicodeCharacter{20AC}{\euro}

% Aufgabe
\newcommand{\task}[1]{\section*{Aufgabe #1}}
\newcommand{\subtask}[1]{\subsection*{#1}}

\begin{document}

\tableofcontents 

\section{Abstract} % (fold)
\label{sec:abstract}

\begin{abstract} 
Dies ist ein CheatSheet for Mathematik für Informatiker im Sommersemester 2016 an der Universität des Saarlands. 

Mitarbeit jeglicher Art ist gerne gesehen. Du hast einen Fehler entdeckt oder ein Thema fehlt? Dann klick hier \url{https://github.com/UdS-Skripte/MFI-II-CheatSheet} und eröffne eine neue \href{https://github.com/UdS-Skripte/MFI-II-CheatSheet/issues/new}{Issue} oder füge dein Thema selber ein mit einem \href{https://github.com/UdS-Skripte/MFI-II-CheatSheet/compare#fork-destination-box}{Pull\ Request}.

\textbf{Bitte hilf mit, dieses Dokument besser zu machen!}
\end{abstract}

% section abstract (end)

\section{Contributors} % (fold)
\label{sec:contributors}

\begin{itemize}
	\item Marvin Hofmann (Author)
\end{itemize}

% section contributors (end)

\newpage

\section{Vektoren} % (fold)
\label{sec:vektoren}

Ein Vektor wird geschrieben: $a = (a_1, \cdots, a_n)^t \in \mathbb{R}^n$ oder
$$a = \begin{pmatrix}a_{1}\\\vdots\\a_{n}\end{pmatrix} \in \mathbb{R}^n$$

\subsection{Vektorformeln} % (fold)
\label{sub:vektorformeln}


Mit $x,y \in \mathbb{R}^n$ und $\lambda \in \mathbb{R}$

\begin{tabular}{|l|p{2.5cm}|p{5cm}|l|}
	\hline
	Bezeichnung & Operator & Definition & Ergebnisbereich\\
	\hline
	Vektoraddition & $x+y $&$
	\begin{pmatrix}
		x_1 + y_1\\
		\vdots\\
		x_n + y_n
	\end{pmatrix}$ & $\in \mathbb{R}^n$\\
	\hline
	Vektormultiplikation &  $ \lambda \cdot y $&$ 
	\begin{pmatrix}
		\lambda \cdot y_1\\
		\vdots\\
		\lambda \cdot y_n
	\end{pmatrix}$ & $\in \mathbb{R}^n$\\
	\hline
	Skalarprodukt & 
	$\langle x,y \rangle $ \newline
	$x \cdot y$
	&$ x_1 \cdot y_1 + \cdots + x_n \cdot y_n$
	& $\in \mathbb{R}$\\
	\hline
	Betrag &
	$ ||x|| $ \newline
	$ ||x||_2 $& 
	$ \sqrt{\sum_{i=1}^n x_i^n} $ \newline
	$ \sqrt{ x_1^2 + \cdots + x_n^2 } $ \newline
	$ \langle x,x \rangle^{ \frac{1}{2} }$
	& $\in \mathbb{R}$\\
	\hline
	Abstand & $d(x,y)$ & $||x-y||$ & $\in \mathbb{R}$\\
	\hline
\end{tabular}

% subsection vektorformeln (end)

\subsection{Eigenschaften des Skalarprodukts} % (fold)
\label{sub:eigenschaften_des_skalarprodukts}

% subsection eigenschaften_des_skalarprodukts (end)

% section vektoren (end)

\end{document}
